\documentclass[12p,a4paper,notitlepage]{scrartcl}
\usepackage[magyar]{babel}
\usepackage[utf8]{inputenc}
\usepackage{t1enc}
\usepackage{amsmath} %matek
\usepackage{amsthm}
\usepackage{amsfonts}
\usepackage{graphicx}
\usepackage{verbatim} %komment miatt
\usepackage{multicol}
\usepackage{struktex}
\usepackage{listings}
\usepackage{sectsty}
\usepackage[mathscr]{euscript}
\usepackage{amssymb}
\usepackage{geometry}
\usepackage{contour}
\usepackage{abc}
\usepackage{enumerate}


 \geometry{
 a4paper,
 total={170mm,257mm},
 left=10mm,
 right=10mm,
 top=15mm,
 }
 

 
 \title{Bullet rigidBody táblázat}
 \subtitle{}
 \date{}
\begin{document}
\maketitle

Egy rigidBody lehet:
\begin{itemize}
\item Dynamic
\begin{itemize}
\item mass > 0.0f
\end{itemize}
\item Static
\begin{itemize}
\item mass == 0.0f
\end{itemize}
\item Kinematic
\begin{itemize}
\item mass == 0.0f
\end{itemize}
\end{itemize}


\begin{tabular}{|p{65mm}|p{85mm}|p{20mm}|}
\hline
Tagfüggvény & Leírás & alapérték \\
\hline 
\hline 
applyForce( \newline
btVector3 $force$, \newline
btVector3 $rel\_pos$) & A $rel\_pos$-ba mér egy $force$ erőt & - \\
\hline
applyCentralForce( \newline
btVector3 $force$) & Tömeggközépontbeli erő hatás & - \\
\hline
applyImpulse( \newline 
btVector3 $impulse$, \newline
btVector3 $rel\_pos$) & Adott $rel\_pos$-ba hat $impulse$ impulzussal. & -\\
\hline
applyCentralImpulse(\newline 
btVector3 $impulse$, \newline 
btVector3 $rel\_pos$) & Értelemszerűen... & - \\
\hline
applyTorque(\newline 
btVector3 $torque$) & Forgatónyomaték adás a testnek. & - \\
\hline
applyTorqueImpulse( \newline
btVector3 $torqueImpulse$) & Forgatónyomaték impulzus adás a testnek. & - \\
\hline
\end{tabular}

\begin{tabular}{|p{65mm}|p{85mm}|p{20mm}|}
\hline
setFriction(\newline
btScalar $frict$) & A test ,,súrlódási együtthatója''. \newline
A valóságban ilyen nem létezik, csak különböző felülettípusú párokra adható együttható.
A bullet a két test együtthatóiból ($frict$ paraméterek) számolja ezt az elméletben létező együtthatót. & 0.5f \\
\hline
setRollingFriction(\newline
btScalar $frict$) & A test gurulásnál létrejövő súrlódásának együtthatója. \newline
Akkor van hatása, ha egy valamilyen guruló test (gömb, henger, stb..) egy felületen gurul. & 0.0f \\
\hline
setAnisotropicFriction(\newline
btVector3 $anisFrict$, \newline
int $frictMode$) & A bullet lehetőséget ad arra, hogy különböző irányokban/tengelyek mentén különböző legyen a súrlódás. Az első paraméter az irányt/tengelyt adja meg a második paraméter a súrlódás típusát: 0 - ansiotropic friction kikapcsolva, 1 - anisotropic friction, 2 - anisiotropic rolling friction & $frictMode = 0$ \\
\hline
setSpinningFriction(\newline 
btScalar $frict$) & Spinning friction akkor keletkezik, ha a test egy felületen a pörög úgy, hogy a forgás tengelye a
a felületnek az érintési pontjában vett normálisa. & 0.0f \\
\hline
setRestitution(\newline
btScalar $rest$) & Coefficient of Restitution. Azt határozza meg, hogy az adott test, valamely másik testtel való ütközés után milyen sebességgel pattanjon vissza a közeledési vektorral ellentétes irányban. Ha pld. $rest == 0$, akkor ,,hozzátapad'' a másik testhez. & 0.0f\\
\hline
setDamping(\newline
btScalar $lin\_damp$, \newline
btScalar $ang\_damp$) & Az adott lineáris vagy anguláris mozgás csillapítása (közegellenállás). 0.0f-től 1.0f-ig állítható. Pld. 0.01f azt jelenti, hogy minden időegységben-ben a sebességvektorból levonódik annak 1\%-a. 1.0f-ra állítva a test nem mozdul. & $lin\_damp = 0$, $\ ang\_damp = 0$ \\
\hline
setLinearFactor(\newline
btVector3 $factor$) & A $factor$ paraméter megadja, hogy a testre ható adott $(x,y,z)$ irányú erő esetén mekkora legyen a testre ható erő az egyes irányokban. Pld., ha $factor = (2.0,0,0)$, akkor a testre $(2 x,0,0)$ erő hat. (=> y,z irányokban fix a pozíciója). & $(1,1,1)$ \\
\hline
setAngularFactor(\newline
btVector3 $factor$) & Ld. linearFactor, csak szögekre. & $(1,1,1)$ \\
\hline
setContactStiffnessAndDamping(\newline
btScalar $stiffness$, \newline
btScalar $damping$) &
A test merevségét állítja be. Minél kisebb a stiffness, erő hatására annál könnyebben nyomható össze a test az ütközések hatására. A damping mondja meg, hogy az ütközést követő összenyomódás során, mennyi legyen a sebesség csillapítása.
Nyilván ez hatással van a restitutionra-ra is. & $stiffness = 1e+18$, $damping = 1.0f$ \\
\hline
setSleepingThreshold(\newline
btScalar $linear$, \newline
btScalar $angular$) &
??? & $linear = 0.8f$, $angular = 1.0f$ \\
\hline






\end{tabular}



\end{document}